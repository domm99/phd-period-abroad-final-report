\documentclass[12pt]{article}
\usepackage{mgates-letter}
\definecolor{dark_blue} {rgb}{0., 0., 0.65}

\usepackage{textcomp}
\usepackage{mathrsfs}  % mathscr font
\usepackage{boxedminipage}
\usepackage{rotating}
\usepackage[inline]{enumitem}
%\usepackage{natbib}
\usepackage{xcolor}
\usepackage[colorlinks, filecolor=dark_blue, urlcolor=dark_blue, linkcolor=black, citecolor=black]{hyperref}

\newcommand{\todoinline}[1]{{\color{violet} #1}}

\begin{document}

\begin{titlepage}

	\newcommand{\HRule}{\rule{\linewidth}{0.5mm}}
	\center
	
	\textsc{\Large Ph.D. Programme in Computer Science And Engineering}\\[0.5cm]
	
	\textsc{\Large XXXIX Cycle}\\[0.6cm]
	
	\hrule width \hsize \kern 1mm \hrule width \hsize height 2pt 
	\vspace{0.8cm}
	{ \large \bfseries Ph.D. Period Abroad Proposal for Marco Polo}\\[0.6cm]
	{ \large Transfer Learning and Coordination Techniques for Distributed Collective Intelligence }\\[0.6cm]

	\bfseries{June, 2025}


    \vspace{1.5cm}
    
    \noindent
    \begin{minipage}[t]{0.3\textwidth}
        \raggedright
        \textbf{Supervisors:}\\[0.5cm]
        Prof. Mirko Viroli\\
        Prof. Danilo Pianini\\
        Prof. Matteo Ferrara
    \end{minipage}%
	\hfill
    \begin{minipage}[t]{0.3\textwidth}
        \centering
        \textbf{Abroad Supervisor:}\\[0.5cm]
        Ivana Dusparic 
    \end{minipage}
    \hfill
    \begin{minipage}[t]{0.3\textwidth}
        \raggedleft
        \textbf{PhD Candidate:}\\[0.5cm]
        Davide Domini
    \end{minipage} \\[0.6cm]

	\hrule width \hsize height 2pt \kern 1mm \hrule width \hsize height 1pt
	\vspace{0.4cm}

\end{titlepage}

\section{Proposal}\label{sec:intro}

\paragraph{Context.}
Computing devices have become ubiquitous in everyday life.
%
This trend has paved the way for research fields aimed at exploiting
 the potential of device collectives to build next-generation systems, 
 including: collective computing~\cite{DBLP:journals/computer/Abowd16}
 and Collective Adaptive Systems (CAS)~\cite{DBLP:journals/sttt/WirsingJN23,robyphdthesis}.
%
More in detail, following Mitchell's definition~\cite{DBLP:conf/metacognition/Mitchell05}, 
 we refer to CAS as distributed systems comprising multiple agents 
 such that each agent:
 \begin{enumerate*}[label=(\roman*)]
	\item can interact with other agents either directly or indirectly;
	\item does not individually posses system-wide knowledge;
	\item can exhibit learning to expand its personal knowledge; and
	\item can make decisions based on collective or aggregated knowledge from some of its peers.
 \end{enumerate*}
%
Notably, we focus on systems involving a large number of agents -- potentially in the hundreds
 or thousands -- which is commonly referred to in the literature 
 as many-agents~\cite{DBLP:phd/ethos/Yang21a}.

In this proposal, we focus on a particularly challenging problem within CAS: 
 enabling scalable decentralized machine learning across a large number of agents. 
% 
A key issue in this setting is how to reuse knowledge across agents operating in 
 heterogeneous and dynamic environments. 
% 
This motivates the exploration of transfer learning as a core strategy, 
 allowing agents to benefit from others' experiences while reducing redundant 
 learning and improving overall convergence.

\paragraph{Opportunities and challenges.}
These systems enable the development of innovative applications in a wide range of real-world domains, such as: 
 smart cities~\cite{DBLP:conf/icse/IftikharRBW017}, 
 traffic control~\cite{DBLP:journals/tits/ChuWCL20,DBLP:books/sp/Muller2011/ProthmannTBHMS11} 
 with autonomous vehicles~\cite{DBLP:journals/corr/BojarskiTDFFGJM16}, 
 coordinated robot swarms for search and rescue~\cite{DBLP:journals/ijon/ZhouLLXS21} 
 or environmental monitoring~\cite{DBLP:conf/acsos/AguzziVE23}, and many more.
%
Nevertheless, while these systems have significant potential, their engineering 
 presents several challenges.
%
First and foremost, control and decision-making demand particular attention.
%
Achieving an optimal balance between centralized and decentralized control is crucial, 
 as neither extreme is feasible nor desirable~\cite{DBLP:conf/coordination/CasadeiPVN19,DBLP:journals/tits/ChuWCL20,DBLP:journals/jair/LyuBXDA23} 
 in many dynamic systems. 
% 
Indeed, excessive centralization may lead to bottlenecks and single points of failure, 
 whereas complete decentralization can hinder coordination and consistency.
%
Additionally, the dynamic nature of these systems, characterized by constant 
 environmental changes, mobility, and potential component failures, requires adaptive 
 learning mechanisms capable of responding swiftly 
 to evolving conditions~\cite{DBLP:journals/swarm/PrasetyoMF19}.
%
Another key consideration is the locality principle, where operational efficiency and cost 
 are heavily influenced by the spatial proximity of data sources, processing units, and users.
%
Furthermore, partial observability introduces uncertainty, as individual components may have 
 limited or incomplete information about the global state, complicating accurate 
 decision-making~\cite{DBLP:conf/uai/HeDB22}.

Transfer learning could provide a principled way to share and reuse knowledge while respecting 
 local constraints. 
%
This opens up research questions around how to coordinate learning, 
 when and what to transfer, and how to ensure stability and efficiency in dynamic 
 multi-agent systems.

\paragraph{Research Gap.}
Despite significant progress in the field of cooperative learning within CAS, 
 several challenges remain open, indicating substantial research gaps.
%
Learning in environments with a high number of agents is often unstable and challenging to manage. 
%
Centralized learning architectures are frequently impractical, primarily due to privacy concerns or technical 
 constraints related to communication and synchronization overheads. 
% 
Although decentralized approaches have been proposed as a viable alternative, they typically lead to 
 suboptimal performance when compared to centralized counterparts, largely due to difficulties in maintaining global 
 coordination and consistency.
%
Another critical challenge arises from the heterogeneous data distribution among agents.
% 
This heterogeneity is frequently influenced by the spatial distribution of the agents.
%
This builds on the assumption that devices in spatial proximity have similar experiences and make similar 
 observations~\cite{esterle2022deep}, as
 the phenomena to capture is intrinsically context dependent.

\paragraph{Research plan.}
This research is expected to involve:
\begin{itemize}
	\item Investigation of scalable transfer learning mechanisms for decentralized learning in many-agent systems;
	\item Investigation of macroprogramming coordination strategies -- such as dynamic leader election -- for enhancing 
	 transfer learning in decentralized many-agent scenarios, particularly where spatial proximity influences 
	 data distribution and learning dynamics;
	\item Integration of the above into simulated environments and evaluation on representative case studies 
	 (e.g., traffic control, robotic coordination).
\end{itemize}

\bibliographystyle{unsrt}
\bibliography{bibliography}

\end{document}
